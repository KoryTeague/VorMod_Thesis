%
% PROJECT: Thesis for ETD, defense
%   TITLE: Approaches to Joint Base Station Selection and Adaptive Slicing in Virtualized Wireless Networks (Working Title)
%  AUTHOR: Kory Teague
% SAVE AS: KT_Thesis.tex
% REVISED: May 10, 2018

\documentclass[12pt,dvips]{report}

\setlength{\textwidth}{6.5in}
\setlength{\textheight}{8.5in}
\setlength{\evensidemargin}{0in}
\setlength{\oddsidemargin}{0in}
\setlength{\topmargin}{0in}

\setlength{\parindent}{0pt}
\setlength{\parskip}{0.1in}

% Uncomment for double-spaced document.
%\renewcommand{\baselinestretch}{2}

% \usepackage{epsf}

\usepackage{cite}

\usepackage[pdftex]{graphicx}
\graphicspath{{Figures/}}
\DeclareGraphicsExtensions{.pdf,.jpeg,.png}

\usepackage{amsmath}
\usepackage{tcolorbox}
\usepackage{subcaption}
\usepackage{amsfonts}
\usepackage{bbm}
\usepackage{color}
\usepackage{setspace}

%\onehalfspacing
\doublespacing

\newcommand\myeq{\mathrel{\overset{\makebox[0pt]{\mbox{\normalfont\tiny\sffamily def}}}{=}}}

\begin{document}

\thispagestyle{empty}
\pagenumbering{roman}
\begin{center}

% TITLE
{\Large 
Approaches to Joint Base Station Selection and Adaptive Slicing in Virtualized Wireless Networks
}

\vfill

Kory A. Teague

\vfill

Thesis submitted to the Faculty of the \\
Virginia Polytechnic Institute and State University \\
in partial fulfillment of the requirements for the degree of

\vfill

Master of Science \\
in \\
Electrical Engineering

\vfill

Allen B. MacKenzie, Chair \\
Luiz DaSilva \\
R. Michael Buehrer \\
Mohammad J. Abdel-Rahman

\vfill

% Date, location of defense?
June 1, 2018 (TBD)\\
Blacksburg, Virginia

\vfill

Keywords: TBD
\\
Copyright 2018, Kory A. Teague

\end{center}

\pagebreak

\thispagestyle{empty}
\begin{center}

{\large Approaches to Joint Base Station Selection and Adaptive Slicing in Virtualized Wireless Networks}

\vfill

Kory A. Teague

\vfill

(ABSTRACT)

\vfill

\end{center}

\iffalse
The need for concrete examples increases when technology becomes
difficult to explain.  In documentation for computer systems
especially, we see a wide audience of field experts attempting to
comprehend documentation for computer software and hardware of which
they should only require a cursory understanding.  Additionally, as
the pace of the information age quickens we see document authors
struggle for \textit{examplia-concretes} with wide applicability, and
consistently rely on excerpts from Shakespearean literature as a
public-domain source for their various explications.

We predict the twenty-first century will be no different.  Actuarial
studies show explosion in the information industry such that four out
of five persons will be \textit{bona fide} electronic document
authors; many of those will have one or more college degrees.  We
prove through computer simulation \textsc{Machinum Simitatores} that
authors of twenty-first century literature will be affected by these
examples and will include metaphor with Shakespearean source into
their writing with increasing frequency.
\fi

\vfill

% GRANT INFORMATION

This work received support in part from the National Science Foundation via work involved with the Wireless @ Virginia Tech research group.

\iffalse
That this work received support from the Southeastern Universities
Research Association (SURA) ``Monticello Library Project'' is purely
coincidental.
\fi

\pagebreak

% Dedication and Acknowledgments are both optional
% \chapter*{Dedication}
% \chapter*{Acknowledgments}

\tableofcontents
\pagebreak

\listoffigures
\pagebreak

\listoftables
\pagebreak

\pagenumbering{arabic}
\pagestyle{myheadings}

\chapter{Introduction} \label{ch:intro}
% The following line is to allow document to publish while it doesn't have any normal citations.  Remove once actual citations are present. 
\textcolor{blue}{\textit{Placeholder to make bibTeX happy:}} \cite{1421931}

\textit{Here I introduce the problem I have been working on (BS selection and dynamic slicing within a VWN).  Start the paragraph talking about the increasing demands within the cellular network; IoT devices, trends toward massive demands.  Mention how costs to maintain and increase the capacity of the network with current costs are rising.  Lead into necessary needs for 5G as intended to be implemented over the next two years, and how virtualization is a potential solution to increase capacity while minimizing cost within 5G networks.}

\section{Literature Review} \label{sec:litreview}

\textit{Discuss the previously existing literature that I have been reading.  Discuss wireless cellular networks, virtualization and their use in wireless networks.  Likely work discussing the network without borders paradigm that is at the core of the work.  Discuss optimization and stochastic optimization.  Discuss metaheuristic algorithms, especially genetic algorithms, and perhaps how they have been used for and to simplify optimization.}

\iffalse
%\markright{Albert J. Kippleby \hfill Chapter 1. Introduction \hfill}

William Shakespeare has profoundly affected the field of literature
worldwide \cite{1421931}.  In the United States there was a surge of Shakespearean
literature starting in the 1960s, with the opening of the Montgomery
Shakespearean festival and continuing into the present ...
\pagebreak

%%%%%%%%%%%%%%%%%
%
% Include an EPS figure with this command:
%   \epsffile{filename.eps}
%

%%%%%%%%%%%%%%%%
%
% Do tables like this:

 \begin{table}
 \caption{The Graduate School wants captions above the tables.}
\begin{center}
 \begin{tabular}{ccc}
 x & 1 & 2 \\ \hline
 1 & 1 & 2 \\
 2 & 2 & 4 \\ \hline
 \end{tabular}
\end{center}
 \end{table}
\fi

\pagebreak
\chapter{VNB Model} \label{ch:vnbmodel}

\textit{Begin breaking down the model used as the basis for the work.  Start with a lead in, then start defining the model in the first subsection.  Expound on some of the definitions and descriptions that I moved past in the conference paper (perhaps describing the SSLT model more fully).  Make sure that the definitions and work are generalized for writing about a constructed VWN (or constructed VWNs, depending on the terminology I intend to use) for multiple RPs and SPs.}

\section{Network Area Definitions} \label{sec:networkdefs}

\textit{Lay out the how the network and area is structured here.  Define the SSLT demand field.  This is effectively section 2 of my previous conference paper.}

\section{Stochastic Optimization} \label{sec:stochopt}

\textit{Now the that model, area, and definitions are defined, proceed to use those terms and define the overall generalized stochastic optimization problem.  This is effectively section 3 (a) of my conference paper, but with, perhaps, a little more focus.}

\pagebreak
\chapter{Approximation Approaches} \label{ch:approaches}

\textit{In this chapter, I work on defining the approximation approaches used in my work.  Lead in to discussing the need to approximate the stochastic optimization problem from section \ref{sec:stochopt} to adequately solve my work, then introduce the two approaches I used to approximate the optimization problem: the DEP/its sampling/generalized post-selection slicing and the genetic algorithm as a selection method.}

\section{Deterministic Equivalent Program} \label{sec:dep}

\textit{Introduce the idea of a DEP as an approach for solving the original stochastic problem.  Present the solved problem here in the form of the true deterministic equivalent program - as in, it is actually an equivalent to the original stochastic problem - with all the necessary expansions and additional variables.  Focus on how this formulation no longer includes any stochastic variables and is purely deterministic.  Mention that the trade off is that the deterministic variables are part of a infinitely large set of potential scenarios.}

\subsection{Sampling Approaches} \label{subsec:dep_sampling}

\textit{As the infinitely large set of scenarios renders the problem unable to be solved, it needs to be sampled into a finite set to be solved.  Present the structure and nomenclature used to imply a sampled set of scenarios, and describe the structure of how the scenarios are sampled into a truncated set.  Might be worth mentioning that there are other methods that might be better for sampling beyond the completely random sampling approach I am using.  Worth consideration?}

\subsubsection{Sample Average Approximation} \label{subsubsec:dep_sampling_saa}

\textit{At what point is the sampling enough?  As the set of scenarios considered within the sampled DEP increases, it more closely compares to the original DEP and the stochastic optimization problem, but it also becomes increasingly difficult to solve as the number of scenarios considered increases.  So, it is beneficial to understand that a certain known number of scenarios provides a reasonably tight - what does reasonable mean? - solution to the original DEP to avoid being unnecessarily computationally expensive to solve.  Finding this minimum necessary number of scenarios can be done via a sample average approximation (SAA) analysis, which should not be too complicated to do.}

\subsection{Adaptive Slicing} \label{subsec:dep_slicing}

\textit{Now that we have a (close) approximation to the DEP and the original stochastic optimization problem, we have a method for deriving the minimum cost BS selection and adaptive slicing for the desired VWN.  However, this selection is overly time consuming to constantly run, and the BSs selected for the VWN(s) by the VNB are fairly constant, so all that is needed is to dynamically (read: adaptively) slice the selected BSs to the various SPs.  To do this, we simplify the sampled DEP such that it has only one scenario - ostensibly, the current scenario in time - and the BSs selected set to be a constant rather than a decision variable.  The resulting problem is a single stage linear program that is much simpler to solve.  This is used to adaptively slice resources to the demand.}

\section{Genetic Algorithm} \label{sec:ga}

\textit{Now that the first approach - DEP and its sampling - has been tackled, and the necessary tool to evaluate it has been derived from it - the simplified adaptive slicing program - move on to the genetic algorithm approach for approximating the BS selection process.  Discuss the core algorithm of a genetic algorithm, then the various approaches that I used in its process (e.g., binary chromosomes, elitism, uniqueness, uniform crossover, bitwise mutation).}


\pagebreak
\chapter{Testing and Simulations} \label{ch:testsim}

\textit{In this chapter I will be introducing four different cases to test the provided approximation approaches.  The first will be the test case used in my conference paper (one SP, with homogeneous resources).  The second will be an expansion of the test case used in my conference paper, but with heterogeneous resources; resources that at least differ in capacity and cost, but}

\section{VWN Construction for a Single SP} \label{sec:onesp}
\subsection{Case I: Homogeneous Urban Cellular Network} \label{subsec:onesp_homres}
\subsection{Case II: Impact of Heterogeneous Resources} \label{subsec:onesp_hetres}
\section{VWN Construction for Multiple SPs} \label{sec:mulsp}
\subsection{Case III: Two Similar Urban Cellular Networks} \label{subsec:mulsp_sim}
\subsubsection{Homogeneous Resources} \label{subsubsec:mulsp_sim_homres}
\subsubsection{Heterogeneous Resources} \label{subsubsec:mulsp_sim_hetres}
\subsection{Case IV: SPs with Specialized Demands} \label{subsec:mulsp_spec}
\subsubsection{Homogeneous Resources} \label{subsubsec:mulsp_spec_homres}
\subsubsection{Heterogeneous Resources} \label{subsubsec:mulsp_spec_hetres}

\pagebreak
\chapter{Conclusions} \label{ch:conc}
%%%%%%%%%%%%%%%%%%%%%%%%%%%%%%%%

% If you are using BibTeX, uncomment the following:
%\thebibliography
%
% Otherwise, uncomment the following:
% \chapter*{Bibliography}

% \appendix

% In LaTeX, each appendix is a "chapter"
% \chapter{Program Source}



\bibliography{KT_Thesis}
\bibliographystyle{IEEEtran}

\end{document}